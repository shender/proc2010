\begin{center}
  \large{\textsc{Исследование вытекания воды из сифона.}}
\end{center}

\textbf{Состав:} Антон Грудкин, Дмитрий Крачун, Михаил Недошивин,
Леонид Цейтлин, Вячеслав Базалий (все --- 9 класс).

\textbf{Цель исследования:} установить и измерить факторы, влияющие на
скорость вытекания воды через шланг от гидроуровня за счет эффекта
сифона.

\textbf{Оборудование:} 2м шланга от гидроуровня, рулетка, секундомер,
2 канистры с водой, маркер. 

\textbf{Методика:} мы искали зависимость скорости вытекания воды от
разности уровней воды. Однако очевидно, что она (скорость) так же
пропорциональна площади поперечного сечения трубки. Далее будет
рассказано, как мы собирали данные для выявления зависимости скорости
вытекания воды из трубки (далее $v$) от разности высот (далее $\Delta h$).

\begin{figure}[h]
  \centering
  \begin{tikzpicture}
    \draw[thick,interface](5.5,0) -- (0,0);
    \draw[very thick] (0.5,0) -- (1,1.5);
    \draw[very thick] (2.5,0) -- (2,1.5);
    % вода
    \draw[fill=blue!20,draw=blue!20] (0.75,1.5) rectangle ++(1.5,2);
    \draw[fill=blue!20,draw=blue!20] (1.4,3) rectangle ++(0.2,2.1);
    \draw[fill=blue!20,draw=blue!20] (1.4,5.1) rectangle ++(2.8,0.2);
    \draw[fill=blue!20,draw=blue!20] (4,5.1) rectangle ++(0.2,-3.6);
    
    \draw[line width=2] (0.5,1.5) -- ++(2,0);
    \draw[thick] (0.75,1.5) -- ++(0,2);
    \draw[thick] (2.25,1.5) -- ++(0,2);
    \draw[thick] (0.75,3.5) -- (1.35,5);
    \draw[thick] (2.25,3.5) -- (1.65,5);
    \draw[thick] (1.4,3) -- ++(0,2.3) -- ++(2.8,0) node[above,midway] {шланг} -- ++(0,-3.8);
    \draw[thick] (1.6,3) -- ++(0,2.1) -- ++(2.4,0) -- ++(0,-3.6);
    \draw[thick,fill=blue!20,draw=blue!20] (3.35,0) rectangle ++(1.5,0.75);
    \draw[thick] (3.35,0) rectangle ++(1.5,2);
    \draw[thick] (3.35,2) -- ++(0.6,1);
    \draw[thick] (4.85,2) -- ++(-0.6,1);
    \draw[thick,pattern=north east lines] (3.8,3) rectangle ++(0.6,0.2);
    
    \draw[blue] (0.5,3.5) -- ++(0.25,0);
    \draw[blue,dashed] (2.5,1.5) -- ++(1.5,0);
    \draw[blue,<->] (0.6,1.5) -- ++(0,2) node[midway,left] {$h$};
  \end{tikzpicture}
\end{figure}

Для начала мы выясним, что именно есть $\Delta h$, так как $\Delta h$
можно мерить и как разницу высот между концами трубки, и как разницу
уровней воды в сосудах. Экспериментально мы выяснили, что $\Delta h$
есть разница уровней воды в сосуде, из которого вода вытекает и концом
трубки. Действительно, если он равен 0, то есть конец трубки совмещен
с уровнем воды в сосуде (из которого вода вытекает), то вода перестает
вытекать. Из этого факта можно так же сделать вывод, что скорость
вытекания воды зависит от $\Delta h$. Затем мы взяли канистру из-под
воды и наполнили её так, чтобы вода доходила до того места, где
канистра сужается, чтобы было легче считать объем вытекшей воды. Далее
заметим, что $\Delta h$ меняется от времени, так как уровень воды в первой
канистре падает. Кстати, вторая канистра не играет роль, так как вода
туда только выливается, то есть её можно убрать, чего не сделаешь с
первой.

\textbf{Обоснование методики:} данная методика оптимальна, так как
проста в исполнении и оптимальна в плане погрешности, однако она не
дает сразу явного ответа. Но ведь зависимость невозможно получить
экспериментально. В эксперименте мы получаем в любом случае дискретный
или упорядоченный набор значений, а значит просто так восстановить
функцию не возможно. Наша методика дает наиболее (на наш взгляд)
удобный и систематизированный (упорядоченный) набор значений.

\begin{center}
  \textbf{Расчёты.}
\end{center}

Пусть скорость вытекания воды (м${}^3$/с) --- $f(\Delta h)$, где
$\Delta h$ --- разность высот между уровнем воды в первом сосуде и
концом трубки. Тогда при переливании воды $\Delta h$ уменьшается.

Пусть сосуд прямоугольный и его высота $h$, тогда перепад высот в
начале $h+\Delta h_1$. Разобьём сосуд на $n$ одинаковых кусочков
(прямоугольных параллелепипедов). ($n$ очень большое, то есть $h/n \ll
h$).

Пусть площадь поперечного сечения стакана $S$. Тогда можно считать,
что скорость вытекания воды на каждом из кусочков одинакова. Объем
любого кусочка $V=S*h/n$. Пусть в $i$-ом кусочке перепад высот $h_i= h
+ \Delta h_1 – i*h/n$, тогда $t_i$ --- время переливания $i$-ого
кусочка, равно $\dfrac{Sh/n}{f{h_i}}$. Значит, 

\begin{equation}
  \label{eq:bz09_2_1}
   T_{\text{общ}} = \sum_{i=1}^{n} \frac{Sh/n}{f(h_i)} \Rightarrow
   v_{\text{ср}} = \frac{Sh}{\sum_{i=1}^{n} \frac{Sh/n}{f(h_i)}} =
   \frac{h}{\sum_{i=1}^n \frac{h/n}{f(h_i)}}. 
\end{equation}

Нарисуем график функции $1/f(x)$ на промежутке от $\Delta h_1$ до
$\Delta h_1+h$ и разобьём фигуру, ограниченную

\begin{itemize}
\item графиком $1/f(x)$,
\item осью $x$,
\item прямой $x= \Delta h_1$,
\item прямой $x= \Delta h_1+h$
\end{itemize}
на $n$ прямоугольников (см. рис.), тогда $S_i$ --- площадь $i$-ого куска $\approx$
$(h/n)/f(h_i)$ (прямоугольник со сторонами $(h/n)$ и $(1/f(hi))$).

Значит, вся  площадь $\approx \sum_{i=1}^n \frac{h/n}{f(h_i)}$. Таким
образом, 
\begin{equation}
  \label{eq:bz09_2_2}
  \int\limits_{\Delta h_1}^{\Delta h_1 + h} \frac{dx}{f(x)} = \sum_{i=1}^n \frac{h/n}{f(h_i)}.
\end{equation}

Тогда для средней скорости

\begin{equation}
  \label{eq:bz09_2_3}
  v_{\text{ср}} = \frac{h}{\sum_{i=1}^n \frac{h/n}{f(h_i)}} \approx
  \frac{h}{\int\limits_{\Delta h_1}^{\Delta h_1 + h} \frac{dx}{f(x)}}.
\end{equation}

При малых перепадах высот ($\Delta h_1 \gg h$) $f(\Delta h)$ можно
считать константой, тогда 

\begin{equation}
  \label{eq:bz09_2_4}
  v_{\text{ср}} \approx f(\Delta h_1).
\end{equation}


% график
\begin{figure}[ht]
  \centering
  \begin{tikzpicture}
    \begin{axis}[xlabel={$t$, с},ylabel={$h$, см},grid=major,height=8cm,compat=newest]
      \addplot[thick,color=red,mark=*,error bars/.cd,x dir=both,y dir=both,x
      fixed=0.2,y explicit] coordinates {
        (0,23) +- (0.2,1)
        (1.5,19.8) +- (0.2,0.8)
        (3,17) +- (0.2,0.7)
        (4.5,15.5) +- (0.2,0.6)
        (6,14) +- (0.2,0.5)
        (7.5,13) +- (0.2,0.4)
        (9,12.5) +- (0.2,0.4)
        (10.5,12.25) +- (0.2,0.3)
      };
      \draw (90,160) node[red] {$f_1$};
      \addplot[thick,color=blue,mark=*,error bars/.cd,x dir=both,y dir=both,x
      fixed=0.2,y explicit] coordinates {
        (0,0) +- (0.2,1)
        (1.5,3.2) +- (0.2,0.8)
        (3,6) +- (0.2,0.7)
        (4.5,7.5) +- (0.2,0.6)
        (6,9) +- (0.2,0.5)
        (7.5,10) +- (0.2,0.4)
        (9,10.5) +- (0.2,0.4)
        (10.5,10.75) +- (0.2,0.3)
      };
      \draw (90,85) node[blue] {$f_2$};
    \end{axis}
  \end{tikzpicture}
\end{figure}

\begin{center}
  \textbf{Измерения.}
\end{center}

\begin{table}[h]
  \centering
  \begin{tabular}{|c|c|c|c|c|c|c|c|c|}
    \hline
    Время & 0 & 1,5 & 3 & 4,5 & 6 & 7,5 & 9 & 10,5\\
    \hline
    Высота & 23 & 19,8 & 17 & 15,5& 14 & 13 & 12,5 & 12,25\\
    \hline
  \end{tabular}
\end{table}

\textbf{Результат:} в расчетах указано как рассчитывать среднюю
скорость вытекания воды с учетом того, что $\Delta h$ меняется
(см. методику). Формула в результате выведена методом подбора
коэффициентов и проверена:

\begin{equation}
  \label{eq:bz09_2_5}
  v = \sqrt{2 \Delta h g}  S \pi g.
\end{equation}

\textbf{Вывод:} мы считаем, что для больших $\Delta h$ данная формула не
работает, так как <<пропускная способность>> трубки ограничена.
%%% Local Variables: 
%%% mode: latex
%%% TeX-master: "../../report"
%%% End: 

