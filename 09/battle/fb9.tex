\begin{center}
 \textsc{Физбой, 9 класс.}
\end{center}
\vspace{0.01cm}
\parindent=0mm

\taskpic{Катушка катится без проскальзывания по горизонтальной
  поверхности, причем скорость конца нити (точка~$A$) горизонтальна и
  равна $v$. На катушку опирается шарнирно закрепленная в точке~$B$
  доска. Внутренний и внешний радиусы катушки равны $r$ и $R$
  соответственно. Определите угловую скорость $\omega$ доски в
  зависимости от угла
  $\alpha$.}{
\begin{tikzpicture}
  \draw[interface,thick] (4,0) -- (0.5,0);
  \draw[thick] (2.25,0.75) circle (0.75);
  \draw[thick,dashed] (2.25,0.75) circle (0.35);
  \draw[thick,dashed] (2.25,0.75+0.35) -- ++(0.65,0);
  \draw (2.9,1.1) -- ++(0.5,0);
  \draw[fill=black] (3.4,1.1) circle (0.03) node [above] {$A$};
  \draw[thick,->] (3.4,1.1) -- ++(0.5,0) node [below,midway] {$v$};
  \draw[blue,->] (2.25,0.75) node[above=-2] {\tiny{$r$}} --
  ++(-0.35,0);
  \draw[blue,->] (2.25,0.75)  -- ++(-60:0.75) node[above=2]
  {\tiny{$R$}};
  \draw[very thick] (0.5,0) node [left=-4] {$B$} -- ++(46.5:3.5);
  \draw[blue] (1,0) arc (0:46.5:0.5);
  \draw[blue] (1.15,0.25) node {$\alpha$};
\end{tikzpicture}
}


\task{На тело, движущееся с постоянной скоростью $\vec{v}$, начинает
  действовать некоторая постоянная сила. Спустя промежуток времени
  $\Delta t$ скорость тела уменьшается в два раза. Спустя еще такой же
  интервал времени скорость уменьшается еще в два раза. Определите
  скорость $v_{\mbox{\textit{к}}}$ тела через интервал времени
  $3\Delta t$ с начала действия постоянной силы.}

\taskpic{Для покоящейся системы, изображенной на рисунке, найдите
  ускорения всех грузов сразу после того, как была перерезана
  удерживающая их нижняя нить. Считать, что нити невесомы и
  нерастяжимы, пружины невесомы, масса блока пренебрежимо мала, трение
  в подвесе отсутствует.}{
\begin{tikzpicture}
  \draw[interface,thick] (1,3.5) -- ++(2,0);
  \draw (2,3.5) -- ++(0,-1);
  \draw[very thick] (2,2.5) circle (0.5);
  \draw[thick] (1.5,2.5) -- ++(0,-0.75);
  \draw[thick] (2.5,2.5) -- ++(0,-0.75);
  \draw[thick] (1.25,1.75) rectangle +(0.5,-0.25) node[left=4,midway] {$m_1$};
  \draw[thick] (2.25,1.75) rectangle +(0.5,-0.25) node[right=5,midway] {$m_3$};
  \draw[spring] (1.5,1.5) -- ++(0,-0.5) node [midway,left] {$k$};
  \draw[spring] (2.5,1.5) -- ++(0,-0.5) node [midway,right] {$k$};
  \draw[thick] (1.25,1) rectangle +(0.5,-0.25) node[left=4,midway] {$m_2$};
  \draw[thick] (2.25,1) rectangle +(0.5,-0.25) node[right=5,midway] {$m_4$};
  \draw (2.5,0.75) -- +(0,-0.5);
  \draw[interface,thick] (3,0.25) -- ++(-0.9,0);
\end{tikzpicture}
}

\task{Вблизи поверхности земли свободно падает тело массой $m$. В
  некоторый момент времени в него попадает (и застревает)
  горизонтально летящая тяжелая пуля массой $M$. Определите время
  падения $t$ тела, если известно, что пуля попала в тело на половине
  пути, а время свободного падения тела с той же высоты равно
  $t_0$. Считать, что масса пули много больше массы тела ($M \gg
  m$). Сопротивлением воздуха пренебречь.}

\taskpic{В ведре находится смесь воды со льдом массой $m = 10\mbox{
    кг}$. Ведро внесли в комнату и сразу же начали измерять
  температуру смеси. Получившаяся зависимость температуры от времени
  $T(\tau)$ изображена на графике. Удельная теплоемкость воды равна
  $c_{\mbox{\textit{в}}} = 4{,}2 \mbox{
    кДж}/(\mbox{кг}\cdot\mbox{К})$, удельная теплота плавления льда
  $\lambda = 340 \mbox{ кДж/кг}$. Определите массу
  $m_{\mbox{\textit{л}}}$ льда в ведре, когда его внесли в
  комнату. Теплоемкостью ведра
  пренебречь.}{
\begin{tikzpicture}
  \draw[blue,dashed] (0,2) -- (3,2) -- (3,0);
  \draw[thick,->] (0,0) -- ++(3.5,0);
  \draw[thick,->] (0,0) -- ++(0,3.5) node[right=3,fill=white] {\tiny{$T,
    {}^\circ C$}};
  \draw[very thick,red] (0,0) -- (2.5,0) --
  ($(2.5,0)!1.3!(3,2)$);
  \foreach \x in {20,40,60} {
    \draw (\x/20,0.1) -- ++(0,-0.2) node[below=-3] {\tiny{\x}};
  }
  \foreach \y in {1,2,3} {
    \draw (0.1,\y) -- ++(-0.2,0) node[left=-3] {\tiny{\y}};
  }
  \draw (3.5/2,-0.5) node {\tiny{$\tau$, мин}};
\end{tikzpicture}
}

\task{В схеме, изображенной на рисунке, сопротивления всех
  резисторов одинаковы и равны $R$. Напряжение на клеммах равно
  $U$. Определите силу тока $I$ в подводящих проводах, если их
  сопротивлением можно
  пренебречь.}

\begin{figure}[h]
  \centering
  \begin{circuitikz}[scale=0.8]
    \draw[thick] (4,7) to [generic] (4,4) to [generic] (7,4) to [generic]
    (4,7);
    \draw[thick] (7,4) to [generic] (4,1) to [generic] (4,4);
    \draw[thick] (4,1) to [generic] (1,4) to [generic] (4,7);
    \draw[thick] (1,4) to [generic] (4,4);
    \draw[o-] (1,2) -- (1,4);
    \draw[o-] (2,1) to[out=0,in=-135] (4,4);
    \draw[<->] (1,2) -- (2,1) node [midway,fill=white] {$U$};
    \draw[fill=black] (1,4)  circle (0.1) node [left] {$A$};
    \draw[fill=black] (7,4)  circle (0.1) node [right] {$B$};
    \draw[fill=black] (4,7)  circle (0.1) node [above] {$C$};
    \draw[fill=black] (4,1)  circle (0.1) node [below] {$D$};
    \draw[fill=black] (4,4)  circle (0.1);
  \end{circuitikz}
\end{figure}

\setcounter{notask}{1}
%%% Local Variables: 
%%% mode: latex
%%% TeX-master: "../../report"
%%% End: 
