\begin{center}
  \textsf{Листок 5.}
\end{center}
\vspace{0.01mm}
\nopagebreak[2]
\task{Автомобиль с работающим двигателем въезжает на обледенелую
  гору, поверхность которой образует угол $\alpha$ с горизонтом. Какой
  высоты гору может преодолеть автомобиль, если его начальная скорость
  при въезде на неё равна $V$, а коэффициент трения колёс о лёд $k$?}

\task{Каков радиус орбиты спутника, лежащей в экваториальной
  плоскости, если тот всё время находится в зените над одной и той же
  точкой земной поверхности?  }

\task{Однородный куб с помощью верёвки, привязанной к середине его
  ребра, подвешен к вертикальной стене. При каких значениях угла
  между верёвкой и стеной куб соприкасается со стеной всей гранью,
  если коэффициент трения его о плоскость равен $k$?}

\task{Лестница опирается на вертикальную стену и пол. При каких
  значениях угла между лестницей и полом она может стоять, если
  коэффициент трения лестницы о пол и о стену равен $k_1$ и $k_2$
  соответственно?}

%%% Local Variables: 
%%% mode: latex
%%% TeX-master: "../../../report"
%%% End: 
