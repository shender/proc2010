\begin{center}
  \textsf{Листок 6.}
\end{center}
\vspace{0.01mm}
\nopagebreak[2]
\task{Необходимо с поверхности земли попасть камнем в цель, которая
  расположена на высоте $h$ и на расстоянии $S$ по горизонтали. При какой
  наименьшей скорости это возможно? Сопротивлением воздуха пренебречь.}

\task{Между целью и миномётом, находящимся на одной горизонтали,
  расположена стена высотой $h$. Расстояние от стены до миномёта $\alpha$, от
  стены до цели $b$. Определить минимальную начальную скорость мины,
  необходимую для поражения цели. Под каким углом при этом надо
  стрелять? Сопротивлением воздуха пренебречь.  }

\task{Гладкий однородный стержень длины $2L$ опирается на край
  полусферической чаши радиуса $R$. Какой угол с горизонтом образует
  стержень в положении равновесия?}

\task{Зенитное орудие может сообщить снаряду скорость $V$ в любом
  направлении. Требуется найти зону поражения, т. е. границу,
  отделяющую цели, до которых снаряд из данного орудия может долететь,
  от недостижимых целей. Сопротивлением воздуха пренебречь.}

\task{Автомобиль удаляется от стены со скоростью $V$ под углом $\alpha$ к
  ней. В момент, когда расстояние до стены равно $l$, шофёр подаёт
  короткий звуковой сигнал. Какое расстояние пройдёт автомобиль до
  момента, когда шофёр услышит эхо? Скорость звука в воздухе $c$.}
