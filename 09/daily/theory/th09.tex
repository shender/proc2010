Теоретический курс для 9 класса был целиком посвящен механике. По
окончании 8 класса школьники имеют определенный объем знаний по этому
разделу физики, и целью курса являлось свести их воедино, а также
осветить вопросы, которым уделяется недостаточное внимание в школьной
программе.

\begin{center}
  \textsf{План занятий.}
\end{center}

\begin{enumerate}
\item Скорость. Радиус-вектор. Мгновенная скорость как предел средней 
скорости. Операции с векторами: сложение, проецирование, скалярное 
произведение.
\item Ускорение. Равноускоренное движение. Скорость и перемещение при 
равноускоренном движении. Движение в однородном поле тяжести: время 
движения, максимальная высота подъема, дальность полета, уравнение 
траектории.
\item Движение по окружности. Период обращения и частота. Угловая 
скорость. Центростремительное ускорение. Нормальное и тангенциальное 
ускорение. Радиус кривизны траектории. Угловое ускорение. Нормальное 
и тангенциальное ускорение при движении с постоянным угловым 
ускорением.
\item Силы. Виды сил. Законы Ньютона. Преобразование координат, 
скоростей и ускорений при переходе в другую систему отсчета. 
\item Закон сохранения импульса. Его связь с третьим законом Ньютона. 
Движение тела с переменной массой.
\item Работа. Кинетическая энергия. Потенциальная энергия. Закон 
сохранения энергии. Потенциальная энергия постоянной силы, силы 
упругости, силы всемирного тяготения.
\item Первый закон Кеплера. Первая и вторая космическая скорость. 
Гравитационный маневр.
\item Центр масс. Второй закон Ньютона для системы тел. Столкновения. 
Рассмотрение абсолютно упругого столкновения в системе отсчета центра 
масс.
\item Правило моментов. Неинерциальные системы отсчета. Сила инерции.
\end{enumerate}


%%% Local Variables: 
%%% mode: latex
%%% TeX-master: "../../../report"
%%% End: 
