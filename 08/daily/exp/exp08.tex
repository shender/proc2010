Занятия экспериментом в 8 классе были направлены на знакомство с общей
культурой выполнения экспериментальных работ. Школьникам была
прочитана вводная лекция о правилах составления отчётов о проведённых
экспериментальных работах, рассказано о погрешностях. По нашим
наблюдениям, на уроках физики в школе погрешностям уделяется
исключительно мало внимания. Тем не менее, для настоящего понимания
экспериментальной работы необходимо уметь свободно пользоваться этим
аппаратом.

Трудность проведения экспериментальных занятий в 8 классе связана ещё
и с тем, что школьники недостаточно хорошо умеют пользоваться
относительно сложным оборудованием. Поэтому одной из задач было
показать школьникам, что даже с помощью тривиального оборудования
(линейка, монетки, нитки) можно достаточно точно померить различные
физические величины. 

\begin{enumerate}
\item Знакомство с трёхмерной системой координат. 
\item Измерение скорости муравья.  \\ \textit{Оборудование:} линейка,
  секундомер, муравей.
\item Измерение массы капли воды. \\ \textit{Оборудование:} ведро,
  стакан, шприц, монетки.
\item Нахождение центра масс палочки. \\ \textit{Оборудование:}
  линейка, монетки, нитки, палочка.
\item Изучение зависимости силы растяжения резинки от её удлинения. \\
  \textit{Оборудование:} резинка, монетки, линейка.
\item Изучение вращательного движения. \\ \textit{Оборудование:}
  камни, нитки. 
\end{enumerate}


%%% Local Variables: 
%%% mode: latex
%%% TeX-master: "../../../report"
%%% End: 