Курс для 7-8 классов был направлен на то, чтобы сформировать у
учащихся представление о физике, как естественной науке. Показать
единство теоретического и экспериментального подхода, логику
построения научного знания. В рамках теоретического курса обсуждались
вопросы описания механических явлений, а экспериментальная часть была
призвана подкрепить теоретические рассуждения на предметно-чувственном
уровне.

\begin{center}
  \textsf{План занятий.}
\end{center}

\begin{enumerate}
\item Введение. Принципы построения естественных наук. Предмет и
  объект физики.  Постулат существования мира.
\item Пространство и его измерение. Начало отсчета. Определение
  положения тела в пространстве. Декартова система координат и
  полярные координаты.
\item Время и его измерение. Приборы для измерения
  времени. Астрономическое и декретное время.
\item Масса. Понятие массы. Измерение массы тел (гравитационная
  масса). Понятие материальной точки.
\item Характеристики механического движения. Траектория, путь,
  перемещение, скорость.
\item Относительность движения. Системы отсчета. Равномерное
  прямолинейное движение. Первый закон Ньютона.
\item Определение силы. Импульс тела. Второй закон
  Ньютона. Инерциальная масса.
\item Движение тела под действием силы. Ускорение. Равноускоренное
  движение. Перемещение при прямолинейном равноускоренном движении.
\item Причины появления силы. Третий закон Ньютона. Парные силы в
  природе.
\item Трение. Виды трения и их особенности. Разные виды трения в
  природе и технике. Закон Кулона-Амантона.
\item Сила упругости. Деформация тел. Виды деформации. Закон
  Гука. Модуль Юнга.
\item Сила тяжести. Сила тяжести вблизи Земли. Ускорение свободного
  падения тел. Закон Всемирного тяготения. Понятие гравитационного
  поля. Графическое представление полей, силовые линии.
  Характеристики поля. Напряженность силы тяжести.
\item Движение планет. Представление о движении планет в разные
  исторические периоды (Птолемей, Коперник, Браге). Законы Кеплера.
\item Движение тела по окружности. Угловая скорость, период,
  частота. Равномерное движение по окружности. Центростремительная
  сила, центростремительное ускорение. Законы Кеплера, как следствия
  закона Всемирного тяготения.
\end{enumerate}


%%% Local Variables: 
%%% mode: latex
%%% TeX-master: "../../../report"
%%% End: 
