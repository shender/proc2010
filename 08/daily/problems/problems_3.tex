\begin{center}
  \textsf{Листок 3.}
\end{center}
\vspace{0.01mm}
\nopagebreak[2]
\task{ Лодка держит курс перпендикулярно берегу и движется со
  скоростью 7,2 км/ч.  Течение относит ее на расстояние 150 м вниз по
  реке. Найдите скорость течения реки и время, затраченное на переезд
  через реку. Ширина реки равна 0,5 км.}

\task{ Пассажир поезда, идущего со скоростью 40 км/ч, видит в течение
  3 с встречный поезд длиной 75 м. C какой скоростью движется
  встречный поезд?  }

\task{ Дельфин плывет со скоростью 18 км/ч вдоль стенок квадратного
  бассейна, описывая квадрат на постоянном расстоянии от прямолинейных
  участков стенок. За 1 мин он полностью обходит бассейн 3 раза. Найти
  расстояние между дельфином и стенкой. Длина каждой стенки 30 м.}

\task{ Пешеход за первые 20 с прошел 30 м, за следующие 40 с ---
  58 м, и за последние 30 с --- 45 м. Определите скорость движения на
  каждом участке и найдите среднюю скорость за все время движения.  }

\task{ На уроке физкультуры Петя и Маша бежали вместе по прямой
  дорожке, стартовав от школы. Затем, Петя побежал быстрее, а Маша
  пошла. Через некоторое время ребята одновременно повернули обратно и
  достигли школы одновременно. Графики зависимости проекции скорости
  ребят на направление дорожки от времени даны на рисунке. Построить
  графики зависимости расстояния между Петей и Машей от времени.  }

\begin{figure}[h]
  \centering
  \begin{tikzpicture}[>=latex]
    \draw[help lines,step=0.5] (0,-0.5) grid (4,4);
    \draw[thick,->] (0,1.5) -- +(3.7,0) node[above=0.4cm,left=-0.2cm,fill=white] {\footnotesize{$t, \unit{мин}$}};
    \draw[thick,->] (0,-0.5) -- +(0,4.5) node[right,fill=white] {\footnotesize{$V_x, \unit{м/c}$}};
    \draw[very thick,red!50!blue] (0,2.5) -- +(0.5,0);
    \draw[very thick,blue] (0.5,3.5) -- +(1,0);
    \draw[very thick,red] (0.5,2) -- +(1,0);
    \draw[very thick,red!50!blue] (1.5,1.5) -- +(0.5,0);
    \draw[very thick,blue] (2,0) -- +(1.5,0);
    \draw[very thick,red] (2,1) -- +(1.5,0);
    \foreach \x in {1,3,5,7}
    \draw (\x/2,1.6) -- +(0,-0.2) node[anchor=north] {\tiny{$\x$}};
    \foreach \y in {-3.75,-1.25,1.25,2.50,5}
    \draw (-0.1,\y*0.4+1.5) -- +(0.2,0) node[left=0.1cm] {\tiny{$\y$}};
  \end{tikzpicture}
  \label{fig:task_08_15}
\end{figure}

\taskpic{ Двое часовых, двигаясь прямолинейно, охраняют с противоположных
  сторон один небольшой объект. Графики зависимости координат часовых
  от времени даны на рисунке. Постройте: а) графики зависимости скорости
  часовых от времени; б) график зависимости скорости первого часового
  относительно второго от времени.}
{
  \begin{tikzpicture}[x=0.75cm,y=0.75cm,>=latex]
    \draw[help lines,step=1] (0,0) grid (4,4);
    \draw[thick,->] (0,-0.2) -- (0,4.5) node[right=0.2cm,fill=white]
    {\footnotesize{$x,\unit{м}$}};
    \draw[thick,->] (0,2) -- (4.6,2) node [above=0.1cm,fill=white] {\footnotesize{$t,\unit{с}$}};
    \foreach \x in {10,20,30}
    \draw (\x/10,2.1) -- +(0,-0.2) node[anchor=north] {\tiny{$\x$}};
    \foreach \y in {-10,10}
    \draw (0,\y/10+2) -- +(0.1,0) node[anchor=west] {\tiny{$\y$}};
    \draw[very thick,red] (0,0) -- (1,1) -- (2,1) -- (3,0) -- (4,2);
    \draw[very thick,blue] (0,4) -- (1,3) -- (2,3) -- (3,4) -- (4,2);
  \end{tikzpicture}
}
