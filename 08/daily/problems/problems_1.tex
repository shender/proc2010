\begin{center}
  \textsf{Листок 1.}
\end{center}
\vspace{0.01mm}
\nopagebreak[2]
\task{ Спортсмен пробегает стометровку за 10 с. Первые 10 м он бежит с
  постоянным ускорением $a$, остальную часть дистанции --- с постоянной
  скоростью $v$. Найдите ускорение $a$ и скорость $v$.  }

\task{ Каретка прибора прошла путь $S$ следующим образом: первую
  половину пути она двигалась с постоянной скоростью $v=12 \unit{м/с}$, вторую
  --- с постоянным отрицательным ускорением так, что в конце пути
  остановилась. Найдите среднюю скорость движения каретки.}

\task{ Ракета стартует с поверхности земли вертикально вверх и в
  течение 10 с поднимается с постоянным ускорением $4{,}9 \unit{м/c}^2$. Затем
  двигатели ракеты отключаются. Найдите максимальную высоту подъема
  ракеты над поверхностью земли. Сопротивлением воздуха
  пренебречь. Ускорение свободного падения считать равным $9{,}8 \unit{м/c}^2$.  }

\task{ Тело свободно падает на землю с некоторой высоты. Начальная
  скорость тела равна нулю. За последнюю секунду оно проходит третью
  часть всего пути. С какой высоты и сколько времени падало тело?
  Сопротивлением воздуха пренебречь. }

\task{ Тело начинает двигаться без начальной скорости с постоянным по
  величине и направлению ускорением $a$. В некоторый момент ускорение
  меняется на противоположное. Найдите путь, пройденный телом за время
  $t_0$ после начала движения, если перемещение тела за время $t_0$ равно
  нулю. }


%%% Local Variables: 
%%% mode: latex
%%% TeX-master: "../../../report"
%%% End: 
