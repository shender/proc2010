\begin{center}
  \textsf{Листок 2.}
\end{center}
\vspace{0.01mm}
\nopagebreak[2]
\task{ Машина, трогаясь с места, проходит путь 1 км с постоянным
  ускорением $0{,}2 \unit{м/c}^2$, затем на отрезке пути 1 км ее ускорение равно
  нулю. После этого машина двигается равнозамедленно до полной
  остановки, затратив на эту часть пути время 40 с. Найдите среднюю
  скорость движения машины.}

\task{ Камень бросают вертикально вверх с начальной скоростью 10 м/с с
  высоты 10 м от поверхности земли. Через 1 с после этого с
  поверхности земли вдоль той же самой вертикали бросают вверх другой
  камень с начальной скоростью 20 м/с. Через какое время и на какой
  высоте они столкнутся?  }

\task{ Из точки, находящейся на высоте $x_0$ над землей, одновременно
  бросают два камня с одинаковыми по величине начальными скоростями 10
  м/с --- один вертикально вверх, второй вертикально вниз. Второй камень
  упал на землю через 2 с после броска. На какой высоте над землей
  находится в это время второй камень? Сопротивлением воздуха
  пренебречь.}

\task{ Для определения постоянной скорости судна относительно воды
  производят пробег судна по прямолинейному участку реки между
  пристанями, расположенными на одном берегу на расстоянии 4,2
  км. Время пробега по течению реки 300 с, а против течения 420.
  Какова скорость судна относительно воды? Скорость течения реки
  постоянна на всем участке испытаний.}

\task{ Велосипедист едет по дороге и через каждые 6 секунд проезжает
  мимо столба линии электропередачи. Увеличив скорость на некоторую
  величину $\Delta v$, велосипедист стал проезжать мимо столбов через каждые 4
  секунды. Как часто он будет проезжать мимо столбов, если увеличит
  скорость еще на такую же величину?}

%%% Local Variables: 
%%% mode: latex
%%% TeX-master: "../../../report"
%%% End: 
