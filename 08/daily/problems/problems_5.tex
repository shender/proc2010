\begin{center}
  \textsf{Листок 5.}
\end{center}
\vspace{0.01mm}
\nopagebreak[2] 

\task{ Кубик плавает в сосуде с водой так, что его верхняя грань
  параллельна поверхности воды. При этом половина кубика погружена в
  воду. Какой слой масла надо долить, чтобы кубик плавал полностью
  погруженным в жидкость, если плотность масла в два раза меньше
  плотности воды и длина ребра кубика равна $a$? Масло с водой не
  смешивается.  }

\task{ В большом сосуде с водой плавает в вертикальном положении
  тонкостенный стакан, в который налито некоторое количество
  воды. Разность уровней воды в сосуде и стакане равна $x$. Как
  изменится эта разность, если в стакан опустить пробку?  }

\task{ Толстостенная лодка с вертикальными стенками и отверстием в дне
  достаточно долго свободно плавает в озере. Затем отверстие снаружи
  затыкают и внутрь лодки опускают бревно. Повысится или понизится
  после этого уровень воды в лодке относительно уровня воды в озере?}

\task{ Шайба массой $M$, имеющая форму цилиндра с площадью основания
  $S$ и высотой $h$, плавает на границе раздела двух несмешивающихся
  жидкостей с плотностями $\rho_1$ и $\rho_2$ ($\rho_1 <
  \rho_2$). Основание шайбы параллельно границе раздела
  жидкостей. Найдите глубину погружения шайбы в нижнюю жидкость.  }
