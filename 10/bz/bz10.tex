\textbf{Цель:} исследовать зависимость силы сопротивления от скорости
тела в воде. 

\textbf{Методика.}

\begin{enumerate}
\item В качестве тела для измерений был выбран шарик для настольного
  тенниса. В него шприцом заливалась вода, что меняло плотность
  шарика.
\item Мячик погружался в воду специально сделанным держателем.
\item В определённый момент мячик отпускали и он поднимался вверх.
\item Вышеописанные действия снимали на видеокамеру с частотой 30
  кадров в секунду. 
\end{enumerate}

\textbf{Физическое обоснование.}

\begin{enumerate}
\item Объём шарика может измерили, обхватив его ниткой. Узнали длину
  окружности и, тем самым, узнали объём.
\item Массу шарика измерили на весах, используя линейку и монеты
  известной массы.
\item Видео, снятое камерой, разбивалось на кадры. Так как около
  ёмкости с водой стояла линейка, то в каждый момент можно было
  определить высоту шарика. Отсюда
  \begin{equation}
    \label{eq:bz_1}
    V = \frac{dh}{dt}, \quad a = \frac{dV}{dt},
  \end{equation}
  где $dt = 1/30$ секунды. Далее, зная ускорение, находим
  \begin{equation}
    \label{eq:bz_2}
    ma = F_a - mg - F_c, \quad F_c = V (g (\rho_{\mbox{ж}} -
    \rho_{\mbox{т}}) - \rho_{\mbox{т}} a). 
  \end{equation}
\end{enumerate}

\textbf{Экспериментальные данные. }

\com{вставить таблицы и график}

\textbf{Теоретическое обоснование. }

Посмотрим на рисунок токов воды во время движения шарика. 

\com{вставить картинку}

В точках 1 и 2 скорость воды $\approx 0$, а в точках 1' и 2'
находящихся чуть дальше от шарика, $V$. Сила вязкого трения, как
известно, равна $F = \eta S V /d$, следовательно, сила сопротивления
линейно зависит от скорости. 

Заметно, что перед шариком образуется <<водяная подушка>>, область
двигающейся вместе с шариком воды (область между точками 3 и
3'). Поток воды отражается от неё, изменяя импульс шарика: $\vec{F}
\Delta t = m \Delta \vec{V}$. 

\com{вставить картинку}

Рассмотрим отдельно взятый поток. Его начальная скорость равна
$\vec{V}_1$, конечная~---~$\vec{V}_2$. Так как площадь сечения воды
уменьшается, то $V_2 > V_1$, но незначительно, поэтому мы будем
считать, что $V_1 = V_2$. 

\begin{equation}
  \label{eq:bz_3}
  \vec{V}_1 + d\vec{V} = \vec{V}_2 \quad \Rightarrow \quad dV = V_1 (2 + 2 \cos
  \alpha). 
\end{equation}

При получении $\Delta \vec{V}_0 = \left(\int \Delta \vec{V}  dS \right)
/S$ перпендикулярные скорости шарика проекции взаимоуничтожаются (в
силу симметрии), а параллельные складываются, но все они
пропорциональны $V_1$, поэтому $\Delta V_0 \sim V_1$. 

Запишем закон изменения импульса: 

\begin{equation}
  \label{eq:bz_4}
  F = \frac{m \Delta V_0}{\Delta t} = \frac{S_{\mbox{ш}} V_1 \Delta t
    \rho \Delta V_0}{\Delta t} = S_{\mbox{ш}} \rho V_1 \Delta V_0 \quad
  \Rightarrow \quad F \sim V_1^2.
\end{equation}

Мы видим, что сила сопротивления пропорциональна квадрату скорости
набегающего потока. 

Кроме того, можно заметить, что на больших скоростях сзади шарика
появляется турбулентность, которая изменяет силу сопротивления. 

%%% Local Variables: 
%%% mode: latex
%%% TeX-master: "../../report"
%%% End: 
