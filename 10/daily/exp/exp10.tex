Экспериментальные занятия в 10 классе были призваны качественно
улучшить имеющуюся у школьников технику проведения экспериментальных
работ. Для этого принципиальные особенности проведения эксперимента
были продемонстрированы на примере распределения Гаусса. Кроме этого,
школьники узнали, что можно экспериментально измерить весьма сложные с
математической точки зрения величины, например, хаусдорфову
размерность природных объектов.

Несколько работ были проведены для того, чтобы продемонстрировать, как
с помощью сравнительно простого оборудования (линейка, секундомер)
можно измерить сложные для прямого измерения величины. 

\begin{enumerate}
\item Измерить модуль упругости каучукового шарика. \\
  \textit{Оборудование:} каучуковый шарик, вода, линейка.
\item Измерить атмосферное давление. \\ \textit{Оборудование:} вода, трубочки, линейка.
\item Распределение Гаусса. \\ \textit{Оборудование:} монетки.
\item Измерение диаметра выходного отверстия у шприца. \\
  \textit{Оборудование:} шприц, линейка.
\item Измерение радиуса песчинки. \\ \textit{Оборудование:}
  секундомер, линейка.
\item Изучение зависимости силы растяжения резинки от её удлинения. \\
  \textit{Оборудование:} миллиметровка, резинка, линейка, грузики.
\item Измерение отношения длин ниток маятника. \\ \textit{Оборудование:} нитки,
  грузики.
\item Нахождение хаусдорфовой размерности листа дерева. \\
  \textit{Оборудование:} лист клёна, линейка.
\item Измерение скорости вытекания воды из крана. \\
  \textit{Оборудование:} линейка.
\item Измерение плотности карандаша. \\ \textit{Оборудование:} вода,
  нитки, линейка, карандаш. 
\end{enumerate}


%%% Local Variables: 
%%% mode: latex
%%% TeX-master: "../../../report"
%%% End: 
