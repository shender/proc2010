В 10 классе лекции были посвящены электростатике. Выбор такого
предмета лекций неслучаен: многолетний опыт работы в жюри физических
олимпиад показывает, что школьники зачастую относятся к электростатике
как к набору простейших формул, не понимая физического смысла
явлений. В школьном курсе физики, по нашим наблюдениям, электростатика
была и остаётся одним из самых сложных разделов.

После окончания курса электростатики был прочитан мини–курс по
гидродинамике. Поскольку математический аппарат был довольно схож с
аппаратом электростатики, удалось разобрать довольно большое число
физических явлений, не выводя заново математические факты, а пользуясь
лишь электромеханическими аналогиями. Такой подход, кроме всего
прочего, хорошо демонстрирует единство методов, применяемых в физике. 

\begin{center}
  \textsf{План занятий.}
\end{center}

\begin{enumerate}
\item История.  Закон Кулона. 2 вида зарядов.  Принцип
  суперпозиции. Независимость парных взаимодействий, дискретность
  заряда. Закон сохранения заряда. Напряженность электрического
  поля. Электрическое поле точечного заряда, системы точечных зарядов.
  Линии векторного поля. Поток вектора электрического поля.
  Электростатическая теорема Гаусса и её доказательство.
\item Свойства линий электрического поля. Теорема Ирншоу и её
  доказательство. Задачи на применение теоремы Гаусса.
  Потенциальность кулоновских сил. Электростатические и гравитационные
  силы. Потенциальная энергия. Потенциал. Потенциал точечного заряда.
\item Производная. Частная производная. Градиент. Связь потенциала с
  напряженностью электрического поля. Краевая задача
  электростатики. Задача Дирихле. Задача Неймана. Краевая задача с
  границей в виде проводника.  Проводники. Определение. Свойства
  проводников. Экранирование.
\item Метод изображений. Решение задач с помощью метода изображений.
  Заряд и заземленная плоскость, заземленных уголок, сфера.
  Электростатическая ёмкость уединенного проводника. Ёмкость шара.
\item Ёмкость конденсатора. Плоский конденсатор. Соединение
  конденсаторов. Энергия взаимодействия зарядов. Энергия
  конденсатора. Электрический ток. Плотность тока. Закон
  Ома. Сопротивление. Соединение сопротивлений.
\item Проводимость. Вывод зависимости сопротивления от проводимости и
  размеров тела. Закон Ома в дифференциальной форме. Правила
  Кирхгофа. Алгоритм и пример решения задач. Закон Джоуля-Ленца и его
  доказательство. Сверхпроводники. Высокотемпературная
  сверхпроводимость.
\item Эффект Пельтье. Энергетические уровни. Эффект
  Томсона. Термопара. Градуировка термопары. Решение схем~---~метод
  контурных токов, метод эквивалентной э.д.с.
\smallskip
\hrule
\item Гидродинамика. Основные определения. Приближения, в которых
  будут выписываться уравнения. Линии тока, их свойства, связь с
  силовыми линиями. Уравнение неразрывности. Поток через малый
  куб. Запись уравнения аналогично теореме Гаусса. Уравнение Бернулли.
\item Вязкость. Формула Пуазейля (распределение скоростей в
  круглой трубе с жидкостью, поток воды через трубу в единицу
  времени).
\item Вязкость или турбулентность? Коэффициент сопротивления и число
  Рейнольдса. Поверхностное натяжение: поверхностный слой, его
  энергия. Оценка радиуса капли в момент отрыва.

\end{enumerate}


%%% Local Variables: 
%%% mode: latex
%%% TeX-master: "../../../report"
%%% End: 
