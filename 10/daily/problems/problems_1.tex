Программа задач по теме «Электростатика» призвана дополнять курс
теоретических лекций. Программа разработана таким образом, что
сложность задач от занятия к занятию постепенно увеличивается. Это
позволяет школьникам начать с простых задач, а затем повысить своё
мастерство до уровня численного описания сложных физических
экспериментов и расчёта некоторых электротехнических
схем. Контрольная, проведённая по итогам темы «Электростатика»,
показала, что слушатели успешно усвоили материал занятий. На последних
занятиях школьники познакомились с гидродинамикой.

Кроме того, на занятиях, прошедших после физбоёв, разбирались задачи,
вызвавшие особое затруднение у школьников. 

Ниже приводятся задачи, обсуждавшиеся на занятиях. Большинство из них
взято из классического задачника повышенной сложности под редакцией
О.Я.~Савченко. 

\begin{center}
  \textsf{Листок 1.}
\end{center}
\vspace{0.01cm}
\nopagebreak[2]
\task{Что будет происходить, если потереть стеклянную палочку о шёлк и
  затем подносить её к бумажному шарику на шёлковой нити? А затем
  коснуться палочкой второго такого же шарика? А затем сблизить
  шарики?}
\task{Как будут меняться показания электроскопа, если трижды коснуться
  его наэлектризованной палочкой?}
\task{Что произойдёт, если соединить два электроскопа: один
  заряженный, а другой --- нет, медной проволокой, держа её в руках? Держа
  её на подвесе из шёлковых нитей? Если соединить те же электроскопы
  чёрной, белой или шёлковой нитью?}
\task{Что будет, если нагреть конец заряженного электроскопа?}
\task{Что будет происходить в последовательном соединении лампочки,
  батарейки и стеклянной трубочки, если последнюю постепенно
  нагревать?}
\task{Что будет, если зарядить один электроскоп стеклянной палочкой,
  потёртой о шёлк, другой --- сургучом, потёртым о шерсть, а затем
  соединить их проволокой?}
\task{Как будет вести себя заряженный груз, подвешенный на нити, если
  приближать к нему руку?}


%%% Local Variables: 
%%% mode: latex
%%% TeX-master: "../../../report"
%%% End: 
