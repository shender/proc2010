\begin{center}
 \textsf{Листок 6.}
\end{center}
\vspace{0.01cm}
\hrule
\parindent=0mm

\task{
  Определите силу, с которой притягиваются друг к другу пластины
  плоского конденсатора, если источник тока, зарядивший конденсатор до
  разности потенциалов 1000 В, отсоединён. Площадь пластин 100
  $\mbox{см}^2$, расстояние между пластинами 1 мм. Изменится ли сила
  взаимодействия пластин, если источник тока будет постоянно
  подсоединён к пластинам?
}

\task{
  Как изменится энергия конденсатора, если при той же разности
  потенциалов между пластинами увеличить все его геометрические размеры
  в $k$ раз?
}

\task{
  На пластины плоского конденсатора помещён заряд $Q$. Площадь пластин
  $S$, расстояние между ними $d$. Какую работу надо совершить, чтобы
  увеличить расстояние между пластинами на $d$? Какую работу надо
  совершить, чтобы сдвинуть пластины на расстояние $x$ относительно
  друг друга? Пластины имеют форму квадрата со стороной $a$. Какая
  совершается работа в обоих предыдущих случаях, если между пластинами
  конденсатора поддерживается батареей постоянная разность
  потенциалов? 
}

