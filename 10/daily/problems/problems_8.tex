\begin{center}
 \textsf{Листок 8 (заключительная контрольная).}
\end{center}
\vspace{0.01cm}
\nopagebreak[2]
\task{
  Два одинаковых заряженных шарика массы $m$, подвешенных в одной
  точке на нитях длины $l$, разошлись так, что угол между нитями стал
  прямым. Определите заряд шариков. 
}

\task{Чему равен поток напряжённости однородного электрического
  поля через боковую поверхность усечённого конуса, радиусы сечения
  которого равны $R$ и $r$? Напряжённость электрического поля $E$
  составляет угол $\alpha$ с осью конуса. }

\task{
  Три проводящие концентрические сферы радиуса $r$, $2r$, $3r$ имеют
  заряд соответственно $q$, $2q$, $-3q$. Определите потенциал на
  каждой сфере.
}

\task{ Какие заряды индуцируются на каждом из трёх одинаковых
  конденсаторов ёмкости $C$, соединённых звездой, если подать на
  свободные концы напряжения 0, $\phi$ и $2\phi$?  }

%%% Local Variables: 
%%% mode: latex
%%% TeX-master: "../../../report"
%%% End: 
