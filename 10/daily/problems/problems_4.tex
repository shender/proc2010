\begin{center}
 \textsf{Листок 4.}
\end{center}
\vspace{0.01cm}
\hrule
\parindent=0mm

\task{Потенциал заряженного проводника 300 В. Какой минимальной
  скоростью должен обладать электрон, чтобы улететь с поверхности
  проводника на бесконечно далёкое от него расстояние?}

\task{
  Заряд 0.1 Кл удалён от заряда 0.2 Кл на расстояние 20 м. Чему равен
  потенциал поля в середине отрезка, соединяющего заряды? 
}

\task{
  Заряды 100, 10, 1, -10, -1, -10 СГС находятся в вершинах правильного
  шестиугольника со стороной 2 см. Чему равен потенциал в центре
  шестиугольника в СИ и СГС? 
}

\task{
  Две бесконечные проводящие изолированные плиты заряжены так, что
  суммарная поверхностная плотность заряда обеих сторон первой плиты
  равна $\sigma_1$, а второй --- $\sigma_2$. Плиты параллельны друг
  другу. Найдите поверхностную плотность заряда на каждой стороне
  плит. 
}

\task{
  Найдите напряжённость электрического поля между тремя пластинами в
  случае, если средняя пластина заземлена. Расстояния между средней
  пластиной и крайними $a$ и $b$. Потенциал крайних пластин $\phi$. 
}

\task{
  В полости металлического шара радиуса $R$ находится заряд
  $Q$. Найдите заряд, индуцируемый этим зарядом на поверхности
  полости. Почему на поверхности шара заряд будет распределён с
  постоянной плотностью? 
}

\task{
  Металлический шар радиуса $R_1$, заряженный до потенциала $\phi$,
  окружают концентрической проводящей незаряженной оболочкой радиуса
  $R_2$. Чему станет равен потенциал шара, если заземлить оболочку?
  Соединить шар с оболочкой проводником? 
}