\begin{center}
 \textsf{Листок 2.}
\end{center}
\vspace{0.01cm}
\hrule
\parindent=0mm

\task{Чему равно электрическое поле на оси равномерно заряженной
  сферы, у которой удалили узкий поясок вблизи экватора?}
\task{Сила взаимодействия между двумя одинаковыми зарядами на
  расстоянии 1 м равна 1 Н. Определите эти заряды в СИ и СГС.}
\task{Предположим, что удалось бы разъединить 1 $\mbox{см}^3$ воды на
  элементарные разноимённые заряды, которые затем удалили бы друг от
  друга на расстояние 100 км. С какой силой притягивались бы эти
  заряды?}
\task{На концах горизонтальной трубы длины $l$ закреплены
  положительные заряды $q_1$ и $q_2$. Найдите положение равновесия
  шарика с положительным зарядом $q$, который помещён внутрь
  трубы. Устойчиво ли это положение равновесия? Будет ли положение
  равновесия отрицательно заряженного шарика устойчивым?}
\task{В атоме водорода электрон движется вокруг протона с угловой
  скоростью $10^{16}$ рад/с. Найдите радиус орбиты.}
\task{Чему равна напряжённость электрического поля в центре равномерно
заряженного тонкого кольца радиуса $R$? Чему она равна на оси кольца
на расстоянии $h$ от центра? Заряд кольца $Q$. }
\task{
  Металлическое кольцо разорвалось кулоновскими силами, когда заряд
  кольца был равен $Q$. Сделали точно такое же новое кольцо, но из
  материала, прочность которого в 10 раз больше. Какой заряд разорвёт
  новое кольцо? 
}