\begin{center}
 \textsf{Листок 7 (гидродинамика).}
\end{center}
\vspace{0.01cm}
\hrule
\parindent=0mm

\task{ Под каким углом к горизонту располагается вода в сосуде,
  движущемся с известным ускорением?  }

\task{ Под каким углом к полу вагона расположится чай в стакане, если
  поезд движется по окружности радиусом 1 км со скоростью 72 км/ч ?  }

\task{ С какой скоростью вытекает вода через маленькую дырочку у дна
  широкой бочки с известным уровнем воды? А из узкой бочки, если
  известны площади бочки и дырки? Что изменится, если дырка не в боку,
  а в дне?}

\task{ Как меняется площадь сечения струи, вытекающей из крана, с
  высотой?  }

\task{ Как определить, в какую сторону действует подъёмная сила на
  крыло известной формы? }

\task{ Даны времена, за которые параллелепипедовидная ванна известных
  размеров наполняется из крана и опустошается через дырку в дне. До
  какой высоты дойдёт вода, если одновременно открыть и кран, и дырку?
}
