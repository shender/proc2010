В 11 классе лекции были посвящены электродинамике. Последние два
занятия были посвящены применению общей теории к анализу электрических
цепей переменного тока, а также кратко изучалась волновая оптика. 

Целью такого курса было изучение разнообразных методов, широко
применяемых в физике. Кроме того, итогом лекций явилось построение
цельной самосогласованной теории, различные следствия которой можно
наблюдать на эксперименте. Фактически, школьники в первый раз
столкнулись с построением работающей физической теории из
первопринципов (в данном случае за основу были взяты законы
сохранения). 

\begin{center}
  \textsf{План занятий.}
\end{center}

\begin{enumerate}
\item Физические принципы электродинамики. Понятие о физических
  полях. Примеры полей: поле температур, поле скоростей. Скалярные и
  векторные поля. Основные операции с векторами. Начала векторного
  анализа. Интегральные теоремы векторного анализа. Теорема Стокса,
  Гаусса–Остроградского.
\item Электростатика. Закон Кулона как следствие теоремы
  Гаусса. Потенциал как следствие безвихревой природы электрического
  поля. Уравнение Пуассона. Теорема Ирншоу.
\item Энергия электрического поля. Аналогии с гидродинамикой (задача
  об обтекании шара идеальной жидкостью) и теплопроводностью
  (распределение температур в среде с простой геометрией).
\item Сохранение энергии в электродинамике. Связь закона сохранения
  энергии и уравнений Максвелла. Получение уравнений Максвелла, их
  физический смысл. Симметрия в уравнениях Максвелла. Калибровка.
\item Частный случай: магнитостатика. Закон Био–Савара–Лапласа, закон
  Ампера. Примеры применения этих законов, расчёт магнитного поля от
  проводников с током простой формы.
\item Физика индукции. Закон Фарадея. Сила Ампера. Задача про
  взаимодействие параллельных токов. Сила Лоренца. Два источника
  возникновения магнитного поля, физический смысл этого.
\item Движущееся электромагнитное поле. Задача про движущуюся
  заряженную плоскость. Понятие об электромагнитной волне. Волновые
  процессы в электродинамике. Волновое уравнение.
\item Задача про конденсатор в переменном электрическом поле. Пример
  решения физической задачи методом последовательных
  приближений. Функция Бесселя.
\item Законы Кирхгофа. Примеры применения законов Кирхгофа к цепям
  постоянного и переменного тока. Катушки индуктивности. Аналогия с
  механикой. Колебательный контур.
\item Понятие о волновой оптике. Явление интерференции как следствие
  волновой природы света. Интерференция от двух точечных источников
  света. Интерференция на плоскопараллельной пластинке.
\end{enumerate}


%%% Local Variables: 
%%% mode: latex
%%% TeX-master: "../../../report"
%%% End: 
