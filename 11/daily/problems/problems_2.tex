\begin{center}
  \textsf{Листок 2.}
\end{center}
\vspace{0.01cm}
\nopagebreak[2]

\task{ В длинной горизонтальной трубе могут скользить без трения два
  поршня, массы которых $M$ и $2M$. Между поршнями находится небольшое
  количество одноатомного идеального газа. Снаружи --- вакуум. В
  начальный момент давление газа $P$, он занимает объем $V$, а поршень
  $M$ имеет скорость $u_{0}$ в направлении второго поршня, который в
  этот момент неподвижен. Найти максимальную скорость тяжелого
  поршня.}

\task{ Вдали от тяготеющих масс в космосе неподвижно висит тонкая
  однородная спица длины $L=10\unit{м}$ и массы $M=1\unit{кг}$. По ней
  может без трения скользить бусинка массы $m=0.1\unit{кг}$. В
  начальный момент бусинка смещена относительно центра спицы на
  $d=1\unit{см}$, система неподвижна.  Через какое время бусинка
  окажется в центре спицы?  Гравитационная постоянная
  $G=6.67 \cdot 10^{-11} \text{ Н}\cdot\text{м}^{2}/\text{кг}^{2}$.}

\task{ На тело, находящееся на горизонтальной шероховатой поверхности
  стола, начинает действовать сила, величина которой возрастает со
  временем по линейному закону. Смещение тела за время $T$, прошедшее
  с момента начала действия силы, составляет $L$, за время $2T$
  смещение равно $50L$. Найти смещение за интервал времени $3T$.  }

\task{ Заряженная частица попадает в среду, где действует сила
  сопротивления, пропорциональная скорости. До полной остановки
  частица проходит путь $S$. Если в среде имеется магнитное поле,
  перпендикулярное скорости частицы, то она при той же начальной
  скорости остановится на расстоянии $L$ от точки входа в среду. На
  каком расстоянии от точки входа в среду остановилась бы частица,
  если бы поле было в два раза меньше? }

\task{ Пластины плоского конденсатора соединены диэлектрической
  пружиной. Начальное расстояние между пластинами $d_{0}$, а после
  того как конденсатор зарядили, оно уменьшилось до величины
  $2d_{0}/2$. Каким будет расстояние между пластинами, если
  параллельно ему подключить такой же конденсатор, но незаряженный.  }
