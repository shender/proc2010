\begin{center}
  \textsf{Листок 3.}
\end{center}
\vspace{0.01cm}
\nopagebreak[2]

\task{ Посередине плоского экрана находится точечный источник света.
  Параллельно экрану расположено плоское зеркало в форме треугольника
  со стороной $a$. Определить площадь зайчика $S$ на экране.  }

\task{ К конденсатору ёмкости $C$ подсоединили резистор сопротивлением
  $R$. Найти зависимость заряда $q(t)$ на резисторе, если в начальный
  момент времени заряд на одной обкладке конденсатора был равен
  $q_{0}$.}

\task{ Найти частоту колебаний $LC$--колебательного контура.  }

\task{ Найти зависимость тока от времени $I(t)$ в $RLC$--контуре, если
  в начальный момент времени заряд на одной обкладке конденсатора был
  равен $q_0$. }

\task{ Из точки А со скоростью $v$ вылетают частицы, имея малый
  разброс $\delta\alpha$, и далее движутся в однородном магнитном поле
  индукции $B$ перпендикулярно ему. Определить, на каком расстоянии от
  точки A соберется пучок, и оцените в этом месте его поперечный
  размер. Масса частиц $m$, заряд $q$. }

\task{ В устройстве для определения изотопного состава ионы калия
  $39K^{+}$ и $41K^{+}$ сначала ускоряются в электрическом поле с
  разностью потенциалов $U$, а затем попадают в однородное магнитное
  поле индукции $B$, перпендикулярное направлению их движения, через
  щель диаметром $d$. Чему должно быть равно $d$, чтобы следы пучков
  на фотопластинке не перекрывались (щель находится в одной плоскости
  с пластинкой). }
