Экспериментальные задачи в 11 классе в большинстве своём состояли из
задач, ранее предлагавшихся на городских и Всероссийских
олимпиадах. Почти в каждой задаче для того, чтобы измерить требуемую
величину с помощью предлагающегося оборудования, школьник должен был
придумать оригинальную методику. Кроме экспериментов по стандартной
школьной программе, в программу 11 класса были включены эксперименты
по интерференции и дифракции (№7, №10), по гидродинамике (№5), а также
по теории колебаний (№9).

\begin{enumerate}
\item Изучение зависимости силы сопротивления среды от скорости. \\
  \textit{Оборудование:} теннисные шарики, нитка, линейка.
\item Изучение распределения вероятностей нахождения груза в
  математическом маятнике. \\ \textit{Оборудование:} нитки, груз.
\item Измерение толщины масляного пятна на поверхности воды. \\
  \textit{Оборудование:} масло, вода, линейка.
\item Измерение массы куска пластилина. \\ \textit{Оборудование:}
  карандаш, вода, нитка, линейка.
\item Измерение радиуса песчинки. \\ \textit{Оборудование:} вода, секундомер.
\item Измерение момента инерции шарика. \\ \textit{Оборудование:}
  каучуковый шарик, вода, линейка.
\item Измерение объёма данных на компакт--диске. \\
  \textit{Оборудование:} компакт--диск, указка, линейка.
\item Измерение плотности карандаша. \\ \textit{Оборудование:} вода,
  нитки, линейка, карандаш.
\item Исследование колебательных мод в двойном маятнике. \\
  \textit{Оборудование:} нитки, монетки.
\item Измерение длины волны лазерной указки. \\ \textit{Оборудование:}
  фольга, булавки, указка, линейка.
\end{enumerate}


%%% Local Variables: 
%%% mode: latex
%%% TeX-master: "../../../report"
%%% End: 