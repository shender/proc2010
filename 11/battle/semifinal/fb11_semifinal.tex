\begin{center}
 \textsc{Физбой, 11 класс. Полуфинал.}
\end{center}
\vspace{0.01cm}
\hrule
\parindent=0mm

\task{Найдите частоту малых колебаний математического маятника
   относительно его нижнего положения равновесия, если непосредственно
   под равновесным положением шарика на расстоянии $h$ от него
   закреплен заряд $Q$. Длина нити $l$, масса шарика $m$, заряд $q$.}

\task{Легкая нерастяжимая нить длиной $2 \mathrm{м}$ удерживается за
   ее концы так, что они находятся на данной высоте рядом друг с
   другом. На нити висит проволочная скобка в виде перевернутой буквы
   U. Масса скобки равна 1 грамму. Нить выдерживает максимальную
   растягивающую силу $F = 5 \mathrm{Н}$. ($F \gg mg$). Концы нити
   начинают перемещать в противоположных горизонтальных направлениях с
   одинаковыми скоростями $1 \mathrm{м/с}$. В какой-то момент нить не
   выдерживает и рвется. На какую максимальную высоту в момент разрыва
   нити взлетит скобка? Сопротивлением воздуха пренебречь.}

\taskpic{Над одним молем идеального газа совершают процесс, показанный на
   рисунке. Найти максимальную температуру газа в течение этого
   процесса (процесс считать квазистатическим)}{\includegraphics[width=4cm]{fb11_s_3.png}}

\taskpic{C одним молем идеального одноатомного газа провели замкнутый
цикл, изображённый на рисунке, где $1–2$ изотерма, $2–3$ изобара,
$3–4$ политропа, для которой $C = R/2$, и $4–1$ изохора. Минимальная
температура, достигаемая газом в цикле, $T_{min} = 300 \,
К$. Политропическим процессом называется процесс, происходящий с
постоянной теплоёмкостью $C$. Определите КПД цикла
$\eta$.}{\includegraphics[width=4cm]{fb11_s_4.png}}

\task{В схеме, изображенной на рисунке, имеются четыре диода. Известно, что при любом
   напряжении, подведйнном к выводам схемы, ток через амперметр не течет. ВАХ трех диодов Д1, Д2 и
   Д3 известны (см. график). Постройте ВАХ четвертого диода.
}

\taskpic{В архиве Снеллиуса найден чертеж оптической схемы. От времени чернила выцвели и на чертеже остались видны только
   три точки - фокус линзы F, источник света S, точка L, принадлежащая плоскости тонкой линзы, и часть прямой линии
   а, соединяющий источник света и его изображение S'. Из пояснений к чертежу следует, что точка S' отстоит от
   плоскости линзы на расстояние, большим, чем точка S.
   Возможно ли по этим данным восстановить исходную схему? Если да, то покажите, как это сделать. Чему равно
   фокусное расстояние линзы?
}{\includegraphics[width=4cm]{fb11_s_6.png}}

\begin{center}
\includegraphics[width=10cm]{fb11_s_5.png}
\end{center}

\setcounter{notask}{1}

% ФИЗБОЙ, СУКА!!!
% 11 КЛАСС

% 1) Электростатика
%    Задача 6.1.14
%    Какой минимальный заряд q нужно закрепить в нижней точне сферической полости радиуса R, чтобы в
%    поле тяжести небольшой шарик массы m и заряда Q находился в верхней точке полости в положении
%    устойчивого равновесия?

%   Замена:
%    Найдите частоту малых колебаний математического маятника относительно его нижнего положения
%    равновесия, если непосредственно под равновесным положением шарика на расстоянии h от него
%    закреплен заряд Q. Длина нити l, масса шарика m, заряд q.

% еще задача 11.7 (Козел)

% 2) Динамика
%    Задача 1.17 (Проволочная скобка)
%    Легкая нерастяжимая нить длиной 2м удерживается за ее концы так, что они находятся на данной высоте рядом друг
%    с другом. На нити висит проволочная скобка в виде перевернутой буквы U. Масса скобки равна 1 грамму. Нить
%    выдерживает максимальную растягивающую силу F = 5Н. (F>>mg). Концы нити начинают перемещать в противоположных
%    горизонтальных направлениях с одинаковыми скоростями 1м/с. В какой-то момент нить не выдерживает и рвется. На
%    какую максимальную высоту в момент разрыва нити взлетит скобка? Сопротивлением воздуха пренебречь.
%    На 10 класс. Если будет совсем уныло - то и на 11ый.

%   Замена:
%    (вроде жесть) 11.84, Козел. На рисунке показана траектория движения лодки, которую
%    оттолкнули от берега рекитак, что в начальнгый момент ее скорость v0 = 1,0 м/с была
%    направлена перпендикулярно берегу. В точке С траектории лодка была через 1 с, в точке D -
%    через 2 с. Определите скорость u течения реки.

% 3) Теплогазы

%    Задача 2.9
%    Над одним молем идеального газа совершают процесс, показанный на рисунке. Найти максимальную температуру газа в
%    течение этого процесса (процесс считать квазистатическим).

%    Экспериментатор Глюк обратил внимание на то, что почти у всех известных ему изопроцессов
%    (изохорического, изобарического, изотермического и адиабатического) график зависимости
%    давления от объема имеет соответствующее название: изохора, ихобара, изотерма и адиабата.
%    У процесса же, в ходе которого не изменяется внутренняя энергия, такого названия нет! Глюк
%    решил восполнить этот пробел и назвал полученную зависимость изоэргой. Далее он решил
%    сравнить ход изоэрги с изотермой и адиабатой для реального одноатомного газа при условиях,
%    близких к нормальным. На рисунке приведены результаты его исследований. Выясните, какому из
%    трех процессов 1-2, 1-3 или 1-4 соответствует "изоэрга", какому - изотерма, а какому -
%    адиабата. Ответ обоснуйте.

%   Замена:
%    Задача 2.7
%    Оцените на какую величину (дельта)х за сутки увеличивается толщина льда, покрывающего водоем, при температуре
%    окружающей среды Т = -20С. В начале похолодания толщина льда была равна 20см. Теплопроводность льда 2.2 Вт/(м*К),
%    его удельная теплота плавления 3.35*10^5 Дж/кг, а плотность 900 кг/см^3.

% 4) Схема

%    Задача 3.7 В схеме, изображенной на рис., имеются четыре диода. Известно, что при любом
%    напряжении, подведйнном к выводам схемы, ток через амперметр не течет. ВАХ трех диодов Д1, Д2 и
%    Д3 известны (см. график). Постройте ВАХ четвертого диода.

%   Замена:

% 5) Оптика

%    11.92, 11.88, 11.54, 11.44!!!!!!!!!!!!




% ---------------------------------------------------------------------------------------------------------------------
% 1) Какой минимальный заряд q нужно закрепить в нижней точне сферической полости радиуса R, чтобы в
%    поле тяжести небольшой шарик массы m и заряда Q находился в верхней точке полости в положении
%    устойчивого равновесия?

% 2) Легкая нерастяжимая нить длиной 2м удерживается за ее концы так, что они находятся на данной высоте рядом друг
%    с другом. На нити висит проволочная скобка в виде перевернутой буквы U. Масса скобки равна 1 грамму. Нить
%    выдерживает максимальную растягивающую силу F = 5Н. (F>>mg). Концы нити начинают перемещать в противоположных
%    горизонтальных направлениях с одинаковыми скоростями 1м/с. В какой-то момент нить не выдерживает и рвется. На
%    какую максимальную высоту в момент разрыва нити взлетит скобка? Сопротивлением воздуха пренебречь.

% 3) Над одним молем идеального газа совершают процесс, показанный на рисунке. Найти максимальную температуру газа в
%    течение этого процесса (процесс считать квазистатическим).

% 4) Оцените на какую величину (дельта)х за сутки увеличивается толщина льда, покрывающего водоем, при температуре
%    окружающей среды Т = -20С. В начале похолодания толщина льда была равна 20см. Теплопроводность льда 2.2 Вт/(м*К),
%    его удельная теплота плавления 3.35*10^5 Дж/кг, а плотность 900 кг/см^3.

% 5) В схеме, изображенной на рис., имеются четыре диода. Известно, что при любом
%    напряжении, подведйнном к выводам схемы, ток через амперметр не течет. ВАХ трех диодов Д1, Д2 и
%    Д3 известны (см. график). Постройте ВАХ четвертого диода.

% 6) В архиве Снеллиуса найден чертеж оптической схемы. От времени чернила выцвели и на чертеже остались видны только
%    три точки - фокус линзы F, источник света S, точка L, принадлежащая плоскости тонкой линзы, и часть прямой линии
%    а, соединяющий источник света и его изображение S'. Из пояснений к чертежу следует, что точка S' отстоит от
%    плоскости линзы на расстояние, большим, чем точка S.
%    Возможно ли по этим данным восстановить исходную схему? Если да, то покажите, как это сделать. Чему равно
%    фокусное расстояние линзы?
% -----------------------------------------------------------------------------------------------------------------------

% Замены:
% 1) Найдите частоту малых колебаний математического маятника относительно его нижнего положения
%    равновесия, если непосредственно под равновесным положением шарика на расстоянии h от него
%    закреплен заряд Q. Длина нити l, масса шарика m, заряд q.

% 2) На рисунке показана траектория движения лодки, которую
%    оттолкнули от берега рекитак, что в начальнгый момент ее скорость v0 = 1,0 м/с была
%    направлена перпендикулярно берегу. В точке С траектории лодка была через 1 с, в точке D -
%    через 2 с. Определите скорость u течения реки.

% 3) Экспериментатор Глюк обратил внимание на то, что почти у всех известных ему изопроцессов
%    (изохорического, изобарического, изотермического и адиабатического) график зависимости
%    давления от объема имеет соответствующее название: изохора, ихобара, изотерма и адиабата.
%    У процесса же, в ходе которого не изменяется внутренняя энергия, такого названия нет! Глюк
%    решил восполнить этот пробел и назвал полученную зависимость изоэргой. Далее он решил
%    сравнить ход изоэрги с изотермой и адиабатой для реального одноатомного газа при условиях,
%    близких к нормальным. На рисунке приведены результаты его исследований. Выясните, какому из
%    трех процессов 1-2, 1-3 или 1-4 соответствует "изоэрга", какому - изотерма, а какому -
%    адиабата. Ответ обоснуйте.

% 4) В чайник с нагревательным элементом мощностью Р = 2200 Вт налили V1 = 1,5л холодной воды и включили
%    его. когда вода закипела, он автоматически отключился. Через t1 = 60с его снова включили, а еще через t2 = 6с
%    вода закипела и чайник выключился. сразу после этого его еще раз включили, но сняв крышку. Автоматический
%    выключатель, срабатывающий под давлением пара, перестал действовать, и вода из чайника начала выкипать. Через
%    t3 = 240с после последнего включения измерили объем оставшейся воды. Он оказался равным V2 = 1,3л. Каково
%    значение удильной теплоты парообразования воды r? Удельная теплоемкость воды с = 4200 Дж/(кг*К), плотность
%    р = 1000 кг/м^3. Теплоемкостью чайника пренебречь.

% 5) Имеется батарейка с ЭДС Е1 и внутренним сопротивлением r1, а также некоторое количество одинаковых батареек
%    с эдс Е2 = Е1/2. Если последовательно с батареей Е1 подключить некоторое количество батареек Е2 и нагрузку, то
%    сила тока в цепи при любом количестве батареек Е2 будет одинаковой. Если же к батарейке Е1 параллельно
%    подсоединить любое число батареек Е2 и ту же нагрузку, то сила тока через нее останется равной прежнему
%    значению. Полярности всех батарей считать одинаковыми. Найдите сопротивление нагрузки R, а также внутреннее
%    сопротивление r2 батареек Е2.

% 6) Говорят, что в архиве Снеллиуса нашли оптическую схему, на которой были изображены линза, предмет - палочка
%    длины l и ее изображение длины l'. От времени чернила выцвели, и остались видны только две точки: вершина
%    палочки S и ее изображение S'. Из текста следовало, что главная оптическая ось проходила через середину палочки
%    перпендикулярно ей. Определите построением положения линзы, главной оптической оси, фокусов линзы, предмета и
%    его изображения и укажите, какая это линза (собирающая или рассеивающая), если l = 5см, l' = 2см, SS' = 15см.
